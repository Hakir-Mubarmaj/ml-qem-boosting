# Placeholder for draft.tex

% Conference paper skeleton for ML-QEM boosting + feature-splitting experiments
\documentclass\[conference]{IEEEtran}
\usepackage{graphicx}
\usepackage{amsmath}
\usepackage{booktabs}
\usepackage{hyperref}
\title{Boosted Trees and Feature Group Ensembles for Machine-Learned Quantum Error Mitigation}
\author{Author Name \\\small Affiliation}
\begin{document}
\maketitle
\begin{abstract}
This work studies the use of boosted tree models and feature-group ensemble strategies for machine-learned quantum error mitigation (ML-QEM). Building on prior ML-QEM methods, we (1) compare boosted models (XGBoost, LightGBM, CatBoost) against Random Forests and neural baselines; (2) analyze mitigation performance as a function of feature set size; and (3) introduce a feature-grouping approach that assigns specialized models to feature blocks and aggregates their predictions. We evaluate our approach in simulation and on IBM hardware, and show that ... (TODO: fill after experiments).
\end{abstract}
\section{Introduction}
\section{Related Work}
\section{Background: ML-QEM}
\section{Method}
\subsection{Feature engineering}
Describe families: global, per-qubit, local density, spectral, noise stats, derived interactions. Explain splitting strategies: by family, by block, by clustering.
\subsection{Model zoo}
Describe models and tuning.
\subsection{Group ensemble and aggregation}
Explain weighted aggregation and stacking with out-of-fold predictions.
\section{Experimental Setup}
Datasets: simulation and hardware. Metrics: RMSE, MAE, runtime. Reproducibility: seeds, release code.
\section{Results}
\subsection{Baseline reproduction}
\subsection{Error vs feature count}
\includegraphics\[width=0.9\linewidth]{results/figures/figure\_error\_vs\_features.png}
\subsection{Group-ensemble results}
Table: best model per bucket (include CSV as table)
\section{Analysis and Ablations}
Feature importances, sensitivity to training size, cross-device generalization.
\section{Conclusion}
\section\*{Acknowledgements}
\appendix
\section{Hyperparameters}
Include ranges and final values.
\bibliographystyle{IEEEtran}
\bibliography{refs}
\end{document}
